\chapter{Theory}

    \section{Collaboration}

    \section{Collaborating in Mixed Reality}
    This section will contain research on the benefits and drawbacks of collaborating in Mixed Reality. As well as the definition of what it means to do so.
    
    \section{Collocated Collaboration}
    
    \section{Remote Collaboration}
    
    \section{Collaborative Learning} % Research on benefits and drawbacks for using VR for collaboration?
    This section will contain research on the benefits and drawbacks of collaborating in VR. As well as the definition of what it means to do so.
    
    \section{Project Based Learning}
    
    \section{Immersion} % Research and definition of what immersion means in the context of VR
    Immersion in relation to computer games is "used to describe the degree of involvement with a game" \cite{Brown2004}. In the paper by Brown in 2014 he defines three different levels of immersion: Engagement, engrossment and total immersion. Each level has its own barriers that needs to be removed for that level of immersion to be possible. Entering a higher level of immersion is correlated with having a higher level of concentration and focus \cite{Jennett2008}. For IVR-Connection this is one of the concepts that can give the user an advantage over collaborating in "real life".
    
    \section{Presence} % Research and definition of what presence means in the context of VR
    Jennet et al. presents two different perspectives on the definition of presence \cite{Jennett2008}. The first has basis in the rationalistic tradition, and defines presence as a psychological sense of being in a virtual environment \cite{Slater1994}. With this perspective the level of presence has to be evaluated through user feedback. The other bases itself on the Heideggerian/Gibsonian metaphysics, and relates presence to the ability of "successfully supported action in the environment" \cite{Zahorik1998}. With this perspective presence can be evaluated through empirical means. Presence and immersion have a lot in common and are often used interchangeably. However, Jennet et al. argues that presence is a state of mind, while immersion is an experience in time \cite{Jennett2008}. With this distinction presence and immersion are allowed to overlap, but it is also possible to have one without the other.
    
    \todo{Explain "Ready-In-Hand" and "throwness". This relates to what presence means for mixed reality}