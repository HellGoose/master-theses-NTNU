\chapter{Introduction}
  
    \section{Motivation}
        With the new rise of Virtual- and Augmented Reality technologies, devices like HTC Vive, Facebook's Oculus, and Microsoft's Hololense and Immersive Headsets have revolutionized the way we see and interact with digital information. These technologies have great potential, not only for entertainment and leisure, but for educational and collaboration purposes as well.
    
        Virtual- and Augmented reality have primarily been used for entertainment in the fledgling years of their resurgence. Since Oculus stated that the Rift was primarily for gaming\cite{OculusMarkZuckerberg}, and HTC's partner, Valve, is one of the biggest names in the games industry\cite{ValveBigName}, the initial focus of the new Virtual reality Devices was clear. Even Augmented Reality technology was made popular by applications like Snapchat\cite{Snapchat} filters and the hugely popular Pokemon Go\cite{PokemonGo}.
        
        Entertainment might be a reason for their initial success, there has also been applications of Virtual- and Augmented Reality outside of entertainment, like helping Norwegian children to learn math\cite{innomag2017} or helping apprentices learn building systems\cite{Nordbohus2017}. These applications show that this kind of technology maybe has an even greater purpose outside of entertainment. With Microsoft's focus on Mixed Reality, giving a platform where Virtual Reality applications and Augmented Reality applications can work together, we wish to explore this technology in the setting of NTNU's Experts in Teamwork
        course.
        
        Both authors attended the Experts in Teamwork(EiT) village "Virtual and Augmented Reality for Learning and Training" during the spring semester of 2017. During this semester we got to experience first hand some of the cutting edge technologies related to virtual and augmented reality. E.g. being able to try the Hololens, a very new and expensive device, making it a rare opportunity.
        
        Four Campuses One Reality is a cross-campus project that aims to put NTNU on the forefront of collaborative innovation. \emph{"The merger of NTNU Trondheim, HIST, Gjøvik University College and Ålesund University College, led to new challenges in supporting collaboration between students and employees across campuses, both within and between cities."} \cite{4C1R-Pitch} Four Campuses One Reality aims to create \emph{"innovative physical and virtual learning arenas and connecting distributed teaching and research groups across NTNU with modern technology."} \cite{4C1R-Pitch} 
                
        In practise the project is looking to create VR-labs across the campuses of NTNU and among other things develop an application (IVR-Connection) that lets students and staff collaborate through virtual reality. This project is currently blooming \cite{4C1R-Geminin} and as a result NTNU has just recently opened two VR labs. \cite{OpenVRLab} 
        
        During the EiT, one of the authors was working on a project that consisted of expanding on a virtual reality collaboration application called IVR-Connection. This application was also at the same time expanded upon by two other groups linked to the EiT village. One of which were located in Gjøvik. The three group worked together on testing and expanding on IVR-Connection by having weekly meetings within the application. Each week some selected participants from each group would enter the virtual world and give status reports, share ideas and test each others solutions. This had great potential for being a productive way of collaborating across large geographical distances.
        
        Experiencing and developing for these technologies during EiT the course is the main reason we chose this theme for our thesis. This makes this yet another reason to explore and expand the concept and idea of IVR-Connection.
    
    \section{Problem Description}
        It turned out however that more often than not it ended up being more work than value. This was partially because there was some technological difficulties with the application and the devices, but mostly because we just were not used to collaborating in a virtual environment. In EiT when a group is not performing optimally, a facilitator can step in and help them achieve greater efficiency. In virtual reality however this proved to be a little bit more challenging. The facilitator had a hard time getting an adequate overview over the collaborative situation in the virtual world, and because he was not co-present in the virtual space, it was also hard for him to relay feedback effectively. Since facilitation had a great effect on our groups efficiency outside of virtual reality, there is precedence for exploring the possibilities for expanding the collaboration application with internal support for facilitation.
        
    \section{Research Question}
    
        \begin{center}
            \begin{tabular}{ p{1cm} p{1.2cm} p{10cm} }
                \textbf{\large{RQ1}} & \multicolumn{2}{ c }{What are the possibilities for supporting collaborative learning and facilitating in the context of EiT in Mixed and Virtual reality?} \\
                \\
                 & \textbf{RQ1.1} & What are the possibilities for supporting and facilitating collaborative learning in the context of project-based learning in VR? \\
                 & \textbf{RQ1.2} & What are the possibilities for supporting and facilitating collaborative learning  in the context of project-based learning in Mixed reality? \\
                 & \textbf{RQ1.3} & What is the the advantages and disadvantages of both approaches?
            \end{tabular}
        \end{center}
        
