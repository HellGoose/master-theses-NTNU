\chapter{Introduction}
    \label{chapter:introduction}
    \section{Motivation}
        With the new rise of Virtual- and Augmented Reality technologies, devices like HTC Vive\cite{vive}, Facebook's Oculus\cite{oculus}, as well as Microsoft's Hololens\cite{hololens} and Immersive Headsets\cite{immersiveheadsets} have revolutionized the way we see and interact with digital information. These technologies have great potential, not only for entertainment and leisure, but for educational and collaboration purposes as well.
    
        Virtual- and Augmented reality have primarily been used for entertainment in the fledgling years of their resurgence. When Oculus stated that the Rifts first focus is gaming\cite{OculusMarkZuckerberg}, and that HTC's partner, Valve, is one of the biggest names in the games industry\cite{ValveBigName}, the initial focus of the new Virtual reality Devices was clear. Even Augmented Reality technology was made popular by applications like Snapchat\cite{Snapchat} filters and the hugely popular game Pokémon Go\cite{PokemonGo}.
        
        Entertainment might be a reason for their initial success, but there has also been applications of Virtual- and Augmented Reality outside of entertainment, like helping Norwegian children to learn math\cite{innomag2017} or helping apprentices learn building systems\cite{Nordbohus2017}. These applications show that this kind of technology might have an even greater purpose outside of entertainment. In addition, the educational system will soon gain even greater focus on educating experts in the field. A project initiated by AR-FOR-EU called Code Reality aims to develop courses for teaching augmented reality in higher level education. \cite{codereality} They estimates to have trails ready as soon as 2019.
        
        As a head start, NTNU has established an Experts in Teamwork(EiT) village called "Virtual and Augmented Reality for Learning and Training". Both authors attended this village during the spring semester of 2017. During this semester we got to experience first hand some of the cutting edge technologies related to virtual and augmented reality. E.g. being able to try the Hololens, a very new and expensive device. This was also where we got introduced to the Four Campuses One Reality project.
        
        Four Campuses One Reality is a cross-campus project that aims to put NTNU at the forefront of collaborative innovation. The project got started because \emph{"the merger of NTNU Trondheim, HIST, Gjøvik University College and Ålesund University College, led to new challenges in supporting collaboration between students and employees across campuses, both within and between cities."} \cite{4C1R-Pitch}  It aims to create \emph{"innovative physical and virtual learning arenas and connecting distributed teaching and research groups across NTNU with modern technology."} \cite{4C1R-Pitch} In practice the project is looking to create VR-labs across the campuses of NTNU and among other things develop an application, named IVR-Connection, which allows students and staff to collaborate through virtual reality. This project is currently blooming \cite{4C1R-Geminin} and as a result, on the 29th of May 2018, two VR labs was officially opened at two of NTNU Trondheim's campuses: Dragvoll and Øya.\cite{OpenVRLab}
        
        During the EiT Course, one of the authors, Stian, worked on a project that consisted of expanding on IVR-Connection. This application was also at the same time expanded upon by two other groups linked to the EiT village. One of which was located in Gjøvik. The three groups collaborated on testing and expanding IVR-Connection by having weekly meetings in the virtual environment provided by the application. Each week, selected participants from each group would enter the virtual world and provide status reports, share ideas, and test each others solutions. This had great potential for being a productive way of collaborating across large geographical distances. Experiencing and developing for these technologies during the EiT course is the main reason we chose this theme for our thesis.
    
    \section{Problem Description}
        At the time, collaborating in virtual reality ended up being more work than value. Partially because there was some technological difficulties with the application and the devices, but mostly because we just were not used to collaborating in a virtual environment. In EiT, when a group is not performing optimally, a facilitator can step in and help them achieve greater efficiency by asking questions to foster further discussion. However, in virtual reality, this proved to be a little bit more challenging. The facilitator had a hard time getting an adequate overview over the collaborative situation in the virtual world, and because he was not co-present in the virtual space, it was also hard for him to relay feedback effectively. Facilitation had a great effect on our groups efficiency outside of virtual reality, thus in this thesis we wanted to further expand the concept of IVR-Connection to better support the cross campus collaborative learning occurring through the EiT village "VR/AR for learning and training".
        
        To achieve a more effective way to facilitate the groups working collaboratively in Virtual Reality, we propose a way to observe and interact with the students in the Virtual Environment through an Augmented Reality platform. This means having the IVR-Connection application, which only supports VR devices, support a cross platform solution with an AR version of IVR-Connection. In the AR version, the facilitators could spectate and interact with the Virtual World the students found themselves in.
        
    \section{Research Question}
        These research questions aim to guide our efforts in exploring solutions for combining augmented- and virtual reality to support and facilitate collaborative learning in the context of EiT.
    
        \begin{center}
            \begin{tabular}{ p{1cm} p{1.2cm} p{8cm} }
                \textbf{\large{RQ1}} & \multicolumn{2}{ p{10cm} }{How to use mixed- and virtual reality to support and facilitate collaborative learning in the context of EiT?} \\
                \\
                 & \textbf{RQ1.1} & How to use virtual reality to support and facilitate collaborative learning in the context of EiT? \\
                 & \textbf{RQ1.2} & How to use mixed reality to support and facilitate collaborative learning  in the context of EiT? \\
                 & \textbf{RQ1.3} & What are the advantages and disadvantages of both approaches? \\
                \\
                \textbf{\large{RQ2}} & \multicolumn{2}{ p{10cm} }{Can our solution be used in other experience-based learning scenarios outside EiT?} \\
            \end{tabular}
        \end{center}
        
