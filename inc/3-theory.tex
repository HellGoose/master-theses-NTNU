\chapter{Theory}
 
    \section{EiT Learning Methods}
    
    \section{Collaborative Learning}
    
        Collaborative learning is, according to Dillenbourg, \emph{"a situation in which particular forms of interaction among people are expected to occur, which would trigger learning mechanisms".} \cite{dillenbourg1999} This definition, though it is very broad, helps expose the critical element of learning collaboratively. Namely triggering the learning mechanisms. Dillenbourg proposes four categories of ways to increase the probability of triggering these mechanisms: Setting up the initial condition, over-specify the 'collaboration' contract with a scenario based on roles, scaffold productive interactions by encompassing interaction rules in the medium, and monitor and regulate the interactions. \cite{dillenbourg1999}
        
        A similar division was proposed in a more recent paper by Lee. It divides collaborative learning into six procedural elements meant to distinguish collaborative learning from other types of small-group learning: Intentional group formation, continuity of group interaction, interdependence between group members, individual accountability, explicit attention to the development of social skills, and instructor as facilitator. \cite{Lee2009}
        
        In Lee's list the instructor is required to act as a facilitator. Which entails that the \emph{"role of the instructor is one of an expert peer or coach, who offers advice, encouragement and clarification while promoting reflective dialogue and critical thinking through the issuing of timely and relevant questions."} \cite{Lee2009} Dillenbourg's last category concerns monitoring and regulating the group interactions. This category suggests, like Lee, that the instructor does not take the role of a tutor, but instead takes the role of a facilitator. Dillenbourg sees this as important \emph{"because the point is not to provide the right answer or to say which group members is right, but to perform a minimal pedagogical intervention (e.g. provide some hint) in order to redirect the group work in a productive direction or to monitor which members are left out of the interaction."} \cite{dillenbourg1999} Thus both Lee and Dillenbourg lists facilitating as a critical component to learning collaboratively.
        
        \subsection{Computer Supported Collaborative Learning}
        \emph{"In the context of CSCL, the external regulator needs specific tools for monitoring the interactions occurring in different places and/or at different times."} \cite{dillenbourg1999}
    
    \section{Facilitation}
        \subsection{Facilitating for CSCL}
        This is what we want to do