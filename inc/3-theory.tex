\chapter{Theory}

    \section{Collaboration}
    
        \subsection{Collaboration supported by technology}
    
        \subsection{Collocated Collaboration}
        
        \subsection{Remote Collaboration}
    
    \section{Collaborative Learning}
        Collaborative learning is, according to Dillenbourg, \emph{"a situation in which particular forms of interaction among people are expected to occur, which would trigger learning mechanisms".} \cite{dillenbourg1999} This definition, though it is very broad, helps expose the critical element of learning collaboratively. Namely triggering the learning mechanisms. Dillenbourg proposes four categories of ways to increase the probability of triggering these mechanisms: Setting up the initial condition, over-specify the 'collaboration' contract with a scenario based on roles, scaffold productive interactions by encompassing interaction rules in the medium, and monitor and regulate the interactions.
        
        A similar and more clearly defined division was proposed by a paper by Lee, with some additions and modifications. It divides collaborative learning into six procedural elements meant to distinguish collaborative learning from other types of small-group learning: Intentional group formation, continuity of group interaction, interdependence between group members, individual accountability, explicit attention to the development of social skills, and instructor as facilitator. \cite{Lee2009}
        
        Intentional group formation entails designing the group based on learning goals and activities, which lines up with Dillenbourg's category of setting up the initial condition. Continuity of group interaction adds a requirement for the group to sustain their discussion and interactions over a substantial or extended period of time. Interdependence between group members entails creating a perception for group members that they are linked in a way that one cannot succeed unless everyone succeeds, which lines up with Dillenbourg's category of over-specify the 'collaboration' contract with a scenario based on roles. Individual accountability adds that the group members need to be accountable for their own performance as well as that of the group. Explicit attention to the development of social skills entails taking deliberate steps to foster social competencies, which lines up with Dillenbourg's category of scaffold productive interactions by encompassing interaction rules in the medium. Instructor as facilitator dictates that the instructor should take the role of a facilitator, which lines up with Dillenbourg's category of monitoring and regulating the interactions.
        
        This indicates that by supporting the six procedural elements proposed by Lee, can be a way to trigger the learning mechanisms that can be accessed by working collaboratively.

        \subsection{Project Based Learning}
        This is how EiT Works
        
        \subsection{Computer Supported Collaborative Learning}
    
    \section{Facilitation}
        \subsection{Facilitating for CSCL}
        This is what we want to do