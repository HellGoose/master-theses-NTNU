\chapter{Results}
 The results where gathered through three phases. Phase one consisted of exploring Unreal Engine and how IVR-Connection for Vive possibly could add support for Hololens. In the second phase the focus changed to exploring the possibility for making IVR-Connection in Unity Engine with support for Hololens. The third phase focused on expanding on phase 2 based on feedback and remaining unfulfilled requirements.
 
    \section{Phase 1 Implementation} % Fixing and updating to 4.18. Exploring toolkits for hololens
    
    \section{Phase 1 Evaluation} % Internal Evaluation, What made us change
    
    \section{Phase 2 Implementation} % Implementation of IVR-Connection in Unity. Based on requirements.
    
    \section{Phase 2 Evaluation}
    The evaluation of phase 2 consisted of one user test with 7 EiT students and 5 facilitators followed by a focus group session with 2 of the facilitators, and a survey for all 12 of them.
    Due to some bugs in the network code, the multiplayer and cooperation aspect fell out of focus, and only the individual features where properly tested. This means that only the concept of the application and the individual features were evaluated.
    
        \subsection{Survey} % Summary of survey data.
        
        \subsection{Mixed Reality Application Concept} % Summary of interview. Relevant to Mixed Reality.
        % TODO: Connect with requirements.
        The same two facilitators didn't get to do a full test in the Mixed Reality version due to some technical difficulties. But both looked at the concept and a demo with the Hololens. The paragraphs below is a summary of what was said and discussed.
        
        The avatar for the Hololens was too big and intrusive. It is important that the facilitators doesn't take all the attention of the room, but rather acts as a fly on the wall. At least when you are observing. When you are about to give feedback, having a feature that temporarily makes you able to acquire the students attention would be nice. Like for example descending down into their world with a gesture or another in-game indicator that make them understand that you want their attention. It is important that this is comfortable for the students
        
        The facilitators suggested that it would be nice to be able to write on the whiteboard while spectating as Hololens. By adding this feature the facilitator could explain both verbally and visually, and interact with what the students have written themselves. Especially useful for doing EiT exercises.
        
        Even less body language to observe in this version. The application had only avatars with a moving body and a rotating head. No indicator for speech and no hands for gesticulating.
        
        The facilitators also suggested adding the ability to import documents and write on them. This to be able to implement some EiT form exercises into the application. One way to do this is by implementing the ability to import images and draw on them. Making the user able to mark of on forms and saving the result for inspection by the facilitators.
        
        \subsection{Vive Application} % Summary of interview. Relevant to Vive application
        % TODO: Connect with requirements.
        Was tested by two facilitators. One with and one without the Vive headset which both had pros and cons. The following paragraphs is a summary of what were said and discussed in this interview.
        
        The VR lab enabled one facilitator to watch the Trondheim side of the collaboration via the screens in the room. This required a lot of concentration and ended up being tiresome. It felt a little like watching TV and thus he felt more distant than usual when trying to facilitate. Watching through the participants eyes he only got to see what they were looking at, and trying to follow 4 screens looking for collaborative elements was hard. Especially since the level of body language he could read were limited, although he could see 3 of the participants in the room with him, their faces where obscured by the Vive headset. The hardware and equipment also posed limitations, especially on the Gjøvik side, since they only had one phone acting as a microphone. This generated a lot of noise, making it somewhat hard to follow conversations. The threshold for giving feedback and suggestions were higher, since his presence were only heard and not seen. The students completed an EiT exercises name "take space, give space". Due to technical and software limitations this took a lot more time in VR than in the real world. Making it more effort than gain. He also noted that using Skype or another VOIP application with a video feed might have been more efficient for this type of exercises, at leas for this version of the IVR-Collaboration.
        
        The other facilitator tried facilitating with the Vive headset and found that using VR can be uncomfortable and nauseating. For example when the headset picks up signals from the wrong base station, it will get confused and throw you around in the world. She also pointed out that it was hard to pick up on body language and was missing facial expressions, as these are signs she usually picks up on when facilitating in the real world. She also experienced that the VR world had a lot of unrelated stimuli. There is always something more interesting to look at, especially the first times you try it. And since the audio is only in their head, the visual stimuli might at times take more of your focus than intended. In other words, it might be harder to pay attention to what people are saying.
        
        Approximately 70\% of communication comes from body language. And the body language the EiT students used were very different from what the facilitators had observed outside of VR. The facilitators speculated that this could be due to another group dynamic when collaborating with Gjøvik in contrast to internally with each other. They also speculated that it could be due to the anonymity of the session. Especially when someone is talking and not referring to anything in the world, the students tended to look at their hands or pay attention to something else, at least visually. Another factor might be that this is an application that is part of a test project that the EiT students didn't choose to use themselves. It has a lot of bugs and technical difficulties so the facilitators speculated that this could give them an "get it over with" mentality and make them not take it seriously.
        
        Avatars and anonymity can be good for racial and gender bias, as neither of them are visually visible in the application. The facilitators agreed that keeping this feature could be good for inclusion and equality. But that it is still important to be able to distinguish between the different participants, and that name tags can be a good candidate for this. The facilitators also agreed that the avatars should match the purpose. For example do not use animals or crazy outfits for a normal meeting session.
        
        Long session in VR can be tiresome and induce nausea for many different reason. One thing the facilitators noted was that they found it strange that the students decided to stand during whole length of the meeting. This might be due to you not being able to see the chair beside you inside the VR world.
    
    \section{Phase 3 Implementation} % Improving IVR-Connection for Unity. Stability, features, optimization. Based on feedback and remaining unfulfilled requirements.
    
    \section{Phase 3 Evaluation}
    The evaluation for phase three consisted of one user test with two VR experts and a facilitator, followed up by a focus group session. In the test the VR experts were told to cooperate to complete a drawing challenge while the facilitator observed and took notes of how they cooperated. The goal of the test was for the facilitator to evaluate if using Hololens for observing cooperation in VR gives any advantages when trying to facilitate.
        
        \subsection{Focus Group Feedback}
        % TODO: Connect with requirements. So probably write about that first
        
        
        
        
        
        
        