\chapter{Methodology}
This chapter will contain the research questions and the description of the methods used to develop and evaluate the application.

    \section{Research Questions}
    These questions aim to explore the possibilities of using Hololens and mixed reality for collaborative learning and facilitating  in the context of EiT. With a focus on how mixed reality supports collaborative learning and how it compares to traditional VR.  % What to write?
    
    Supporting collaborative learning might be too general, I suggest narrow it down to EiT. Important to include facilitating explicitly since it is a big part of what you are testing!
    RQ: What are the possibilities for supporting collaborative learning and facilitating in the context of EiT in Mixed and Virtual reality?
    RQ1: What are the possibilities for supporting collaborative learning and facilitating in the context of EiT in VR?
    RQ2: What are the possibilities for supporting collaborative learning and facilitating in the context of EiT in Mixed reality?
    RQ3: What is the difference/the advantages and disadvantages of both approaches?
    
    \begin{center}
        \begin{tabular}{ p{1cm} p{1.2cm} p{10cm} }
            \textbf{\large{RQ1}} & \multicolumn{2}{ c }{Is there any advantages to using hololens and mixed reality for collaboration and learning?} \\
            \\
             & \textbf{RQ1.1} & What are the possibilities for supporting collaborative learning in traditional VR? \\
             & \textbf{RQ1.2} & What are the possibilities for supporting collaborative learning in mixed reality? \\
             & \textbf{RQ1.3} & What are the differences?
        \end{tabular}
    \end{center}


    \section{Software Development} % Describe how we implemented it. Requirements + survey + exploration + reevaluation.
    This section will contain a description of what methods were used for developing the application.
    
    \section{Software Evaluation} % Describe how we tested and evaluated the software. Relation to requirements + internal evaluation + user tests + focus groups + survey
    This section will contain a description of what methods were used for evaluating the application. Might be merged with the Software Development section.


% Evaluation qualitative + quantitative
% Could also mention Design-based research: http://edutechwiki.unige.ch/en/Design-based_research