\chapter{Background}
\label{chap:background}
% Explain what this chapter is all about!
In this chapter we will give an introduction to background for our master thesis as well as the tools, technology and previous research that made it all possible. 

\section{Concepts}
    % Chapter intro outlining Our background and general motivation
    This section will take a look at the background for our project. What ideas laid the foundation for our motivations into this field.
    
    \subsection{Virtuality Continuum}
    The Virtuality Continuum(VC), as shown in Figure \ref{fig:virtualcontinuum}. "Is a concept which relates the mixture of classes of objects presented in any particular display situation."\cite{Milgram1994} This article, "\emph{A Taxonomy of Mixed Reality Visual Displays}" by Paul Milgram and Fumio Kishino\cite{Milgram1994} introduces the concept of a "virtuality continuum" to describe the range of environments shown on any particular display. The articles focus on the taxonomy of Mixed Reality displays, and is also one of the earliest adaptors of the concept of Mixed Reality.
    \begin{figure}[h!]
        \centering
        \includegraphics[scale=1]{figures/virtualcontinuum.png}
        \caption{Simplified representation of a "virtuality continuum".\cite{Milgram1994}}
        \label{fig:virtualcontinuum}
    \end{figure}
    
    \subsubsection{Virtual Reality}
    Virtual Reality(VR) is the concept of a virtual space where the user is fully immersed in a virtual world, usually through a Head Mounted Display(HMD). This means that the environment, and everything the user sees and interacts with is completely synthetic. The environment can emulate the real world and seem like reality, be it fiction or otherwise; or this virtual environment might be a world where our physical laws do not apply. On the virtuality continuum, as shown in Figure \ref{fig:virtualcontinuum}, this type of environment resides on the furthest extreme, opposite the real environment.

    \subsubsection{Augmented Reality}
    Augmented Reality(AR) is the concept of a real environment with digital elements superimposed, enhancing the users perception of reality.This is achieved by rendering these digital "\emph{holograms}"\cite{wdc-holo} on a transparent display which the user sees through(e.g. Microsoft's Hololens or Magic Leap. The same effect can also be achieved by superimposing the digital elements onto video captured by a camera in real-time.
    
    \subsubsection{Mixed Reality}
    As shown in Figure \ref{fig:virtualcontinuum}, a Mixed Reality(MR) display can reside anywhere between the extremes of the virtuality continuum. 
    % What is the latest tech and research

\section{Technology}
% What technology are we using.
In this section we will cover the technology that made everything possible. This master is based on cutting edge technology from Microsoft and Unity.

    \subsection{Universal Windows Platform}
    % About UWP. What is it and why we use it
    Universal Windows Platform (UWP) is a platform-homogeneous application architecture created by Microsoft and first introduced in Windows 10. The purpose of this software platform is to help develop universal apps that run on Windows 10, Windows 10 Mobile, Xbox One and Hololens without the need to be re-written for each. It supports Windows app development using C++, C\#, VB.NET, and XAML. The API is implemented in C++, and supported in C++, VB.NET, C\#, F\# and JavaScript. Designed as an extension to the Windows Runtime platform first introduced in Windows Server 2012 and Windows 8, UWP allows developers to create apps that will potentially run on multiple types of devices. %Wikipedia. Probably cite.
    
    \subsection{Mixed Reality Toolkit}
    % What is it and why we use it
    The Mixed Reality Toolkit is a collection of scripts and components intended to accelerate development of applications targeting Microsoft HoloLens and Windows Mixed Reality headsets. The project is aimed at reducing barriers to entry to create mixed reality applications and contribute back to the community as we all grow. %Copy pasta from github readme

    \subsection{Unity}
    % What is it and why we use it
    Unity is a game engine for creating 2D, 3D, VR, AR and MR games and apps. It has its own graphics engine and a full-featured editor that enables you to create games, and deliver your content to virtually any media or device. Unity also features services like cloud building, multiplayer network, version control and analytics.

    More than an engine, Unity helps you achieve ongoing success. It offers everything you need to develop quality content, boost your productivity, and connect with your audience. Tools and resources include the Unity Asset Store, Unity Cloud Build, Unity Analytics, Unity Ads, Unity Everyplay, and Unity Certification. Unity Technologies serves millions of registered developers including large publishers, indie studios, students and hobbyists around the globe.
    
    The leading global game industry software
    Unity plays an important part in a booming global games market. More games are made with Unity than with any other game technology. More players play games made with Unity, and more developers rely on our tools and services to drive their business.
    
    As you can see below, millions of people are creating with Unity every day. This has resulted in global gamers downloading Unity-made games to nearly 2 billion unique mobile devices for Q1 2016 alone.
    
    Unity is also at the forefront of the growing VR market. An estimated 90\% of Samsung Gear VR games and 53\% of Oculus Rift (games at launch) were Made With Unity. %Copy pasta from Unity about page.
    
    \subsection{Unreal}
    One of the oldest and most used game engines (20 years). Been used by multiple award winning AAA games. Written in C++ and is highly optimized for PC, VR and Mobile platforms.
    Visual coding in C++ using blueprints.

    \subsection{Hololens}
    The HoloLens is a head-mounted display unit connected to an adjustable, cushioned inner headband, which can tilt HoloLens up and down, as well as forward and backward. To wear the unit, the user fits the HoloLens on their head, using an adjustment wheel at the back of the headband to secure it around the crown, supporting and distributing the weight of the unit equally for comfort, before tilting the visor towards the front of the eyes.

    In the front is much of the sensors and related hardware, including the cameras and processors. The visor is tinted; enclosed in the visor piece is a pair of transparent combiner lenses, in which the projected images are displayed in the lower half. %copy pasta Wikipedia

    \subsection{Immersive Headsets}
    Under the Mixed Reality Moniker, the Immersive Headsets are VR headsets that uses built-in inside-out tracking. With no need for external sensors, and only one cable for connection with the PC; One can enjoy VR from anywhere. 
    Headsets are provided by multiple different big name retailers, all providing their own designs and solutions for the platform.

\section{Related Work}
    \subsection{Four Campuses One Reality}
    % Something about the vision and idea behind this initiative
    It's about connection Gløs, Draggis, Gjøvik og Ålesund through VR.
    Notes: NTNU merge, VR EiT village, Joint effort between Gjøvik and Trondheim, Idea to connect the campuses through VR, the idea is that, given the right tools, this should work better than using VOIP and video, its all about presence and the tools that can be given to you in a virtual world.

    \subsection{IVR-Connection}
    % About this program and how it fits in with the vision and idea from 4C1R
    Program made by Niclas to test cooperation in VR.
    
    \subsection{EiT VR Village}
    A lot here, including course description, own experiences, interview with LAs!