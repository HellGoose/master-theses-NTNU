\chapter{Methodology}
This chapter will contain the research questions and the description of the methods used to develop and evaluate the application.

    \section{Research Questions}
    These questions aim to explore the possibilities of using Hololens and mixed reality for collaborative learning and facilitating in the context of project-based learning. With a focus on how mixed reality supports collaborative learning and how it compares to traditional VR.  % What to write?
    
    \begin{center}
        \begin{tabular}{ p{1cm} p{1.2cm} p{10cm} }
            \textbf{\large{RQ1}} & \multicolumn{2}{ p{13cm} }{How to use mixed- and virtual reality to support and facilitate collaborative learning in the context of EiT?} \\
            \\
             & \textbf{RQ1.1} & How to use virtual reality to support and facilitate collaborative learning in the context of EiT? \\
             & \textbf{RQ1.2} & How to use mixed reality to support and facilitate collaborative learning  in the context of EiT? \\
             & \textbf{RQ1.3} & What are the advantages and disadvantages for both approaches? \\
            \\
            \textbf{\large{RQ2}} & \multicolumn{2}{ p{13cm} }{Can our solution be used in other experience-based learning scenarios outside EiT?} \\
        \end{tabular}
    \end{center}


    \section{Software Development} % Describe how we implemented it. Requirements + survey + exploration + reevaluation.
    an iterative software development approach (consisting of X phases, each of which included Req-design- implementation-evaluation)
    
    \section{Software Evaluation} % Describe how we tested and evaluated the software. Relation to requirements + internal evaluation + user tests + focus groups + survey
    This section will contain a description of what methods were used for evaluating the application. Might be merged with the Software Development section.


% Evaluation qualitative + quantitative
% Could also mention Design-based research: http://edutechwiki.unige.ch/en/Design-based_research