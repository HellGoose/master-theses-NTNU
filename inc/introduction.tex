\chapter{Introduction}
This chapter will contain a short introduction to the master thesis. What kind of section and subsection could this contain?

\com{EIT should be described in much more detail in INtro, background and when connected to the requirements. 
1.Use official course information 'Guide for students in Experts in Teamwork 2017/2018' What are the learning goals you want to support? The motivation for introducing Mixed reality is that the use of LA that something that was probably not supported well enough in pure VR?
Also this is something nobody tried before. 
Describe also more explicit own experience with Gjøvik EiT spring 2017, identifying what EiT activities were supported in VR and what not. How this led to RQ?}


    
    \section{EiT Virtual and augmented reality (VR/AR) for learning and training}
    \com{A lot here, including course description, own experiences, interview with LAs!}
    
    On NTNU's page about EiT's VR Village (Translated to English):
    \emph{"Virtual and augmented reality (VR/AR) have had an explosive development the last years. This opens up big opportunities in an educational context. For example, safely and easily experience highly realistic educational situations that is not easily accessible in real life, like exploring the seafloor, burning buildings, or war zones. With this technology the students get access to visualizations and simulations that can contribute to a better understanding of syllabus and to better acquiring of skills, e.g. in a virtual lab or operation room.}
    
    \emph{Virtual reality also opens up new opportunities for remote education were students can work together in a virtual hospital or cooperate with students at NTNU Gjøvik and Ålesund. There are being developed games like Pokemon GO to give an introduction to local history. VR/AR can be used in adult education for everything from emergency response training and training of neurosurgeons to work training under direction from NAV. Virtual reality is also used for therapy and medical treatment, e.g. for mental and physical training and stimuli, for disabled people and the elderly."} \cite{EiTVRLandsby}
    
    dawd
    \subsection{Four Campuses One Reality}
    % Something about the vision and idea behind this initiative
    It's about connection Gløs, Draggis, Gjøvik og Ålesund through VR.
    Notes: NTNU merge, VR EiT village, Joint effort between Gjøvik and Trondheim, Idea to connect the campuses through VR, the idea is that, given the right tools, this should work better than using VOIP and video, its all about presence and the tools that can be given to you in a virtual world.
    \subsection{Personal Experience/Using IVR-Connection during EiT}
    
