\chapter{Introduction}
With the new rise of Virtual- and Augmented Reality technologies, devices like HTC Vive, Facebook's Oculus, and Microsoft's Hololense and Immersive Headsets have revolutionized the way we see and interact with digital information. These technologies have great potential, not only in entertainment, but for educational and collaboration purposes as well.

Virtual- and Augmented reality have primarily been used for entertainment in the fledgling years of their resurgence. Since Oculus stated that the Rift was primarily for gaming\cite{OculusMarkZuckerberg}, and HTC's partner, Valve, is one of the biggest names in the games industry\cite{ValveBigName}, the initial focus of the new Virtual reality Devices was clear. Even Augmented Reality technology was made popular by applications like Snapchat\cite{Snapchat} filters and the hugely popular Pokemon Go\cite{PokemonGo}.

Entertainment might be a reason for their initial success, there has also been applications of Virtual- and Augmented Reality outside of entertainment, like helping norwegian children to learn math\cite{innomag2017} or helping apprentices learn building systems\cite{Nordbohus2017}. These applications show that this kind of technology maybe has an even greater purpose outside of entertainment. With Microsoft's focus on Mixed Reality, giving a platform where Virtual Reality applications and Augmented Reality applications can work together, we wish to explore this technology in the setting of NTNU's Experts in Teamwork course.

\com{EIT should be described in much more detail in INtro, background and when connected to the requirements. 
1.Use official course information 'Guide for students in Experts in Teamwork 2017/2018' What are the learning goals you want to support? The motivation for introducing Mixed reality is that the use of LA that something that was probably not supported well enough in pure VR?
Also this is something nobody tried before. 
Describe also more explicit own experience with Gjøvik EiT spring 2017, identifying what EiT activities were supported in VR and what not. How this led to RQ?}


    
    \section{EiT VR Village}
        \subsection{What is EiT?}
        Experts in Teamwork(EiT) is a compulsory subject for Master students at NTNU. There are about 80 villages, each with its unique village theme. The general theme for the villages are problem areas from society and working life. In each village, students are divided into groups and are to define their own project related to the village theme.\cite{EiTAbout}
        
        The point of EiT is for students to develop interdisciplinary teamwork skills. By composing each EiT village of students from a wide range of disciplines, each student will learn how to work together in interdisciplinary groups. There is also a focus on reflection of one's own contribution in a team, and reflection on the team altogether.\cite{EiTAbout}
    
        \subsection{VR/AR for learning and training}
        This thesis has a strong connection to the EiT village "Virtual and Augmented Reality for Learning and Training". This village already had most of the required resources in place for the students to use, and the course description and goal of the village aligns with our motivation for exploring the possibilities in this field.
        
        On NTNU's page about the village (Translated to English): \emph{"Virtual and augmented reality (VR/AR) have had an explosive development the last years. This opens up big opportunities in an educational context. For example, safely and easily experience highly realistic educational situations that is not easily accessible in real life, like exploring the seafloor, burning buildings, or war zones. With this technology the students get access to visualizations and simulations that can contribute to a better understanding of syllabus and to better acquiring of skills, e.g. in a virtual lab or operation room.}
        
        \emph{Virtual reality also opens up new opportunities for remote education were students can work together in a virtual hospital or cooperate with students at NTNU Gjøvik and Ålesund. There are being developed games like Pokemon GO to give an introduction to local history. VR/AR can be used in adult education for everything from emergency response training and training of neurosurgeons to work training under direction from NAV. Virtual reality is also used for therapy and medical treatment, e.g. for mental and physical training and stimuli, for disabled people and the elderly."} \cite{EiTVRLandsby}
    
        \subsection{Our Experience}
        Both authors attended the EiT village "Virtual and Augmented Reality for Learning and Training" during the spring semester of 2017. During this semester we got to experience first hand some of the cutting edge technologies related to virtual and augmented reality. E.g. being able to try the Hololens, a very new and expensive device, making it a rare opportunity. Experiencing and developing for these technologies during the course is the main reason we chose this theme for our thesis.
        
        During the course, one of the authors was working on a project that consisted of expanding on a virtual reality collaboration application called IVR-Connection. This application was also at the same time expanded upon by two other groups linked to the EiT village. One of which were located in Gjøvik. The three group worked together on testing and expanding on IVR-Connection by having weekly meetings within the application. Each week some selected participants from each group would enter the virtual world and give status reports, share ideas and test each others solutions. This had great potential for being a productive way of collaborating across large geographical distances.
        
        It turned out however that more often than not it ended up being more work than value. This was partially because there was some technological difficulties with the application and the devices, but mostly because we just were not used to collaborating in a virtual environment. In EiT when a group is not performing optimally, a facilitator can step in and help them achieve greater efficiency. In virtual reality however this proved to be a little bit more challenging. The facilitator had a hard time getting an adequate overview over the collaborative situation in the virtual world, and because he was not co-present in the virtual space, it was also hard for him to relay feedback effectively. Since facilitation had a great effect on our groups efficiency outside of virtual reality, there is precedence for exploring the possibilities for expanding the collaboration application with internal support for facilitation.
        
    \section{Four Campuses One Reality}
    % Something about the vision and idea behind this initiative
    Four Campuses One Reality is a cross-campus project that aims to put NTNU on the forefront of collaborative innovation. \emph{"The merger of NTNU Trondheim, HIST, Gjøvik University College and Ålesund University College, led to new challenges in supporting collaboration between students and employees across campuses, both within and between cities."} \cite{4C1R-Pitch} Four Campuses One Reality aims to create \emph{"innovative physical and virtual learning arenas and connecting distributed teaching and research groups across NTNU with modern technology."} \cite{4C1R-Pitch} 
    
    In practise the project is looking to create VR-labs across the campuses of NTNU and among other things develop an application (IVR-Connection) that lets students and staff collaborate through virtual reality. This project is currently blooming \cite{4C1R-Geminin} and as a result NTNU has just recently opened two VR labs. \cite{OpenVRLab} This makes this yet another reason to explore and expand the concept and idea of IVR-Connection.
    
    
